\documentclass{sig-alt-release2}
\usepackage{url}
\usepackage{color}
\usepackage{graphics,graphicx}

\usepackage{epsfig}
\usepackage{epstopdf}

\usepackage{colortbl}
\usepackage{multirow}
\usepackage{booktabs}
\usepackage{ifthen}  

\begin{document}
\newcommand{\todo}[1]{\textcolor{red}{#1}}
\def\newblock{\hskip .11em plus .33em minus .07em}

\conferenceinfo{DIM3} {2012, Glasgow, UK} 
\CopyrightYear{2012}
\clubpenalty=10000
\widowpenalty = 10000

\title{{Where in the BOB?}}

\numberofauthors{4}
\author{
\alignauthor
Andrei Palade, 0907350\\
Andrei Mustata, 0907390\\
Cristian Urlea, 1102465\\
Wei Zhang, 1104733\\
          \affaddr{Group F, DIM3}\\
    %  \affaddr{Dim3}\\
      \affaddr{Student no.s}\\
             \email{\{ 0907350, 0907390, 1102465, 1104733\}
             @students.glasgow.ac.uk }
}

\maketitle

\begin{abstract}

\textbf{Where in the BOB?} is a simple web application that enables users to
find the location of a room in the Boyd Orr Building. 
   
This application aims to provide a simple interface where the user inserts the
name of a room in a text field and as a result he gets a floor map in \texttt{SVG}
format with the room highlighted. It provides a list of room names in case of
an ambiguous identifier and the ability to give comments on a particular room. 

\end{abstract}

\section{Aim of Application}

The application name is ``Where in the BOB'' and its purpose is to provide
students, lecturers and visitors of University of Glasgow with an identifying
system of rooms located in Boyd Orr Building. The application intends to
offer a customize interface to this facility where a particular user has the
options to view from a floor plan perspective the location of a given room,
to view details about the room searched and provide comments and ratings of
the room in cause. Moreover, the application will offer details if the room
or the floor is accessible or not.

The assumption, in this case, is to build such a system based on the provided
floor plans of the building. Furthermore, the search facilities are restricted
to a given number of most well know room names amongst students (e.g. Level 3
Computer Science lab) as well as their official names(e.g. Room 720). In this
case, if the room is identified, the result of the search will contain a floor
plan with the location of the searched room highlighted, details about the
room (e.g. number of seats, other details of access, comments on the room),
otherwise it will return a message to confirm that the search process has
failed to identify that particular room (i.e. The rooms was not identified).

\subsection{Goals}
Another goal is to provide the user with a list of room names in case that
there are multiple rooms across campus especially in the Boyd Orr Building.
A particular goal which will provide the user with a consistent and optimized
results is to:
\begin{itemize}
	\item  search again in case of not found 
	\item  leave feedback
	\item  comment on room 
\end{itemize}


\subsection{Assumptions}

We made a number of assumptions based on the purpose of this web application. 
It is assumed that users have a minimum knowledge of English and have the basic 
knowledge of how to use a web browser as well as having Internet connection on 
the used devices.

The visual interaction with the application makes it necessary for the user to 
be able to see and to use one hand in order to interact with the device as they 
do with other applications. This include holding a mouse in order to move a 
pointer around the screen or using fingers to point to different parts of the 
screen on touchscreen handheld devices.

Another assumption we made is that users will be students or visitors with 
minimal knowledge about the campus and they use their handheld device for 
coordination around campus.



\subsection{Constraints}

\subsubsection{Development constraints}

The major constraint we are concerned about is the time necessary to develop 
the application. We will use the available time during the labs to implement 
the required functionality of our system and we will also make use of Django 
web framework for fast development. Due to budgetary constraints we are forced 
to use open source software which cause another issue in terms of technical 
knowledge regarding the tools used.

\subsubsection{Application constraints}
\begin{itemize}
	\item \textbf{End-user environment}: The end-user system can be deployed
	on a wide range of devices (e.g. desktop or mobile systems). These devices
	come with a restricted screen size.
	\item \textbf{Hardware}: 
	\item \textbf{Network}:
	\item \textbf{Interface/Protocol}:
\end{itemize}




\subsection{Functionality}
The required and desired functionality of our application is as follows:
\begin{itemize}
	\item The required functionality:
	\begin{itemize}
		\item Accept textual search terms and return results for a searched
		room identifier 
		
		\item Provide a graphical view of the room that is being searched
		within Boyd Orr Building
		
		\item Return an SVG format of the floor plan where the searched
		room is located with the room highlighted
		
		\item Return a list of matching results for ambiguous defined
		identifiers
		
		\item Allow users to leave comments for a the searched room		
	\end{itemize}

	\item The desired functionality:
	\begin{itemize}
		\item Allow users to see the floor map directory
		\item Allow users to search give feedback on the application
		\item Allow users to return and search for a different room
		\item Allow users to check the accessibility of the room
	\end{itemize}
\end{itemize}

\section{Client Interface}


\section{Application Architecture}

\begin{tabular}{| p{2cm} | p{2cm} | p{3cm}|}
\hline
Data name & Data type & Rationale \\
\hline
id & Integer & Unique ID to identify each floor \\
\hline
description & String & Room description used to supply extra details to optimize the search\\
\hline
date\_created & Date & The floor plan was added\\
\hline
date\_updated & Date & The floor plan was last updated\\
\hline
level & Integer & Optimize the search and reduce the fetch time\\
\hline
rating & Integer & Optimize the search by using popularity\\
\hline
GUID & Integer & Unique ID for the comment\\
\hline
\end{tabular}	


\section{Message Parsing}


\section{Summary and Future Work}
\begin{itemize}
\item	Summary of application and its current state.
\item	Include a list or table of all the technologies, standards, and protocols that will be required.
\item	What are the limitations?
\item Plans for future development
\end{itemize}

\section{Acknowledgements}
Our thanks to the lecturers and demonstrators for their comments and suggestions. And our thanks to the peer reviewers for their feedback.
Be sincere and be specific about how others have helped your group.

\bibliographystyle{abbrv}
\bibliography{sig-proc}

\end{document}
